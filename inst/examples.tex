\documentclass[]{tufte-handout}

% ams
\usepackage{amssymb,amsmath}

\usepackage{ifxetex,ifluatex}
\usepackage{fixltx2e} % provides \textsubscript
\ifnum 0\ifxetex 1\fi\ifluatex 1\fi=0 % if pdftex
  \usepackage[T1]{fontenc}
  \usepackage[utf8]{inputenc}
\else % if luatex or xelatex
  \makeatletter
  \@ifpackageloaded{fontspec}{}{\usepackage{fontspec}}
  \makeatother
  \defaultfontfeatures{Ligatures=TeX,Scale=MatchLowercase}
  \makeatletter
  \@ifpackageloaded{soul}{
     \renewcommand\allcapsspacing[1]{{\addfontfeature{LetterSpace=15}#1}}
     \renewcommand\smallcapsspacing[1]{{\addfontfeature{LetterSpace=10}#1}}
   }{}
  \makeatother
\fi

% graphix
\usepackage{graphicx}
\setkeys{Gin}{width=\linewidth,totalheight=\textheight,keepaspectratio}

% booktabs
\usepackage{booktabs}

% url
\usepackage{url}

% hyperref
\usepackage{hyperref}

% units.
\usepackage{units}


\setcounter{secnumdepth}{-1}

% citations

% pandoc syntax highlighting

% longtable

% multiplecol
\usepackage{multicol}

% strikeout
\usepackage[normalem]{ulem}

% morefloats
\usepackage{morefloats}


% tightlist macro required by pandoc >= 1.14
\providecommand{\tightlist}{%
  \setlength{\itemsep}{0pt}\setlength{\parskip}{0pt}}

% title / author / date
\title{Demonstrating dtkbook Markup}
\author{Daniel Kaplan}
\date{2016-10-29}


\begin{document}

\maketitle




\section{Book markup}\label{book-markup}

I particularly like to create Tufte-style books. The \texttt{tufte} and
\texttt{tint} R packages are vergy helpful here, but they don't do
everything that's needed.

\begin{itemize}
\item
  Marginal notes. In \texttt{tint} and \texttt{tufte}, use the
  \texttt{margin\_note()} function to create a marginal note. Put it in
  an inline chunk, like this:
  \texttt{margin\_note("This\ goes\ in\ the\ margin.")}
  \marginnote{This goes in the margin.}

  Unfortunately, it's not possible via this mechanism to create a
  marginal note that has R content: you can't put an inline chunk inside
  an inline chunk.

  The \texttt{dtkbook::margin\_content()} function works just like
  \texttt{margin\_note}, but allows you to put in an inline chunk in the
  character-string argument. It will be escaped with \texttt{@} rather
  than the back-quote. For instance, here is a simple calculation in a
  marginal note.\marginnote[0cm]{Three plus seven is 10}
\end{itemize}

How is this line formatted?

Another example involves a table like that shown in the margin:
\marginnote[0cm]{Let's say 
\begin{tabular}{r|r}
\hline
mpg & cyl\\
\hline
21.0 & 6\\
\hline
21.0 & 6\\
\hline
22.8 & 4\\
\hline
21.4 & 6\\
\hline
18.7 & 8\\
\hline
\end{tabular}

}

\subsection{Try out the margin\_table()
function}\label{try-out-the-margin_table-function}

Can we put this in the margin?\marginnote[-3cm]{hello 
 
\begin{tabular}{l|r|r}
\hline
  & mpg & cyl\\
\hline
Mazda RX4 & 21.0 & 6\\
\hline
Mazda RX4 Wag & 21.0 & 6\\
\hline
Datsun 710 & 22.8 & 4\\
\hline
Hornet 4 Drive & 21.4 & 6\\
\hline
Hornet Sportabout & 18.7 & 8\\
\hline
\end{tabular} after}



\end{document}
